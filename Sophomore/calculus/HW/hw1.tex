\documentclass{article}
\usepackage[heading=true]{ctex}
\usepackage{amsmath}
\usepackage{amsthm}
\usepackage{amssymb}
\usepackage[a4paper, left=1.5cm, right=1.5cm, top=2.5cm, bottom=2.5cm]{geometry}

\title{HW1}
\author{李昊伦\ 经22-计28\ 2022011545}

\begin{document}
\maketitle
\begin{enumerate}
    \item \begin{enumerate}
        \item[(1)] \begin{proof}[证明]
            函数 $f(x)$ 的 Fourier 级数为\begin{equation}
                f(x)\sim\frac{a_0}{2}+\sum_{n=1}^{\infty}(a_n\cos nx+b_n\sin nx)
            \end{equation}
            其中\begin{equation}
                a_n=\frac{1}{\pi}\int_{-\pi}^{\pi}f(x)\cos nx dx,\quad b_n=\frac{1}{\pi}\int_{-\pi}^{\pi}f(x)\sin nx dx,\quad n\in \mathbb{N}
            \end{equation}
            而\begin{equation}
                \hat{f}(n)=\frac{1}{2\pi}\int_{-\pi}^{\pi}f(x)e^{-inx}dx
            \end{equation}
            因此\begin{equation}
                \frac{a_0}{2}=\frac{1}{2\pi}\int_{-\pi}^{\pi}f(x)dx=\hat{f}(0)
            \end{equation}
            \begin{equation}
                \begin{aligned}
                    a_n
                    &=\frac{1}{\pi}\int_{-\pi}^{\pi}f(x)\cos nx dx \\
                    &=\frac{1}{2\pi}\int_{-\pi}^{\pi}f(x)(e^{-inx}+e^{inx})dx \\
                    &=\hat{f}(n)+\hat{f}(-n)
                \end{aligned}
            \end{equation}
            \begin{equation}
                \begin{aligned}
                    b_n
                    &=\frac{1}{\pi}\int_{-\pi}^{\pi}f(x)\sin nx dx \\
                    &=\frac{i}{2\pi}\int_{-\pi}^{\pi}f(x)(e^{-inx}-e^{inx})dx \\
                    &=i(\hat{f}(n)-\hat{f}(-n))
                \end{aligned}
            \end{equation}
            综上, 得证.
        \end{proof}
        \item[(2)] \begin{proof}[证明]
            \begin{equation}
                \hat{f}(n)=\frac{1}{2\pi}\int_{-\pi}^{\pi}f(x)e^{-inx}dx
            \end{equation}
            \begin{equation}
                \begin{aligned}
                    \hat{f}(-n)
                    &=\frac{1}{2\pi}\int_{-\pi}^{\pi}f(x)e^{inx}dx \\
                    &=\frac{1}{2\pi}\int_{\pi}^{-\pi}f(-x)e^{-in(-x)}d(-x) \\
                    &=\frac{1}{2\pi}\int_{-\pi}^{\pi}f(-x)e^{-inx}dx \\
                    &=\frac{1}{2\pi}\int_{-\pi}^{\pi}f(x)e^{-inx}dx \quad\text{($f(x)$是偶函数)}\\
                    &=\hat{f}(n)
                \end{aligned}
            \end{equation}
            而由(1)知, \begin{equation}
                \begin{aligned}
                    f(x)
                    &\sim\hat{f}(0)+\sum_{n\geq1}(\hat{f}(n)+\hat{f}(-n))\cos nx+i(\hat{f}(n)-\hat{f}(-n))\sin nx \\
                    &=\hat{f}(0)+\sum_{n\geq1}2\hat{f}(n)\cos nx
                \end{aligned}
            \end{equation}
            因此 $f$ 的 Fourier 级数是余弦级数.
        \end{proof}
        \item[(3)] \begin{proof}[证明]
            \begin{equation}
                \begin{aligned}
                    \hat{f}(-n)
                    &=\frac{1}{2\pi}\int_{-\pi}^{\pi}f(x)e^{inx}dx \\
                    &=\frac{1}{2\pi}\int_{\pi}^{-\pi}f(-x)e^{-in(-x)}d(-x) \\
                    &=\frac{1}{2\pi}\int_{-\pi}^{\pi}f(-x)e^{-inx}dx \\
                    &=\frac{1}{2\pi}\int_{-\pi}^{\pi}-f(x)e^{-inx}dx \quad\text{($f(x)$是奇函数)}\\
                    &=-\hat{f}(n)
                \end{aligned}
            \end{equation}
            而由(1)知, \begin{equation}
                \begin{aligned}
                    f(x)
                    &\sim\hat{f}(0)+\sum_{n\geq1}(\hat{f}(n)+\hat{f}(-n))\cos nx+i(\hat{f}(n)-\hat{f}(-n))\sin nx \\
                    &=\sum_{n\geq1}2i\hat{f}(n)\sin nx
                \end{aligned}
            \end{equation}
            因此 $f$ 的 Fourier 级数是正弦级数.
        \end{proof}
        \item[(4)] \begin{proof}[证明]
            $\forall n=2k+1, k\in\mathbb{N}$
            \begin{equation}
                \begin{aligned}
                    \hat{f}(n)
                    &=\frac{1}{2\pi}\int_{-\pi}^{\pi}f(x)e^{-inx}dx \\
                    &=\frac{1}{2\pi}\int_{-\pi}^{0}f(x)e^{-inx}dx+\frac{1}{2\pi}\int_{0}^{\pi}f(x)e^{-inx}dx \\
                    &=\frac{1}{2\pi}\int_{0}^{\pi}f(x-\pi)e^{-in(x-\pi)}d(x-\pi)+\frac{1}{2\pi}\int_{0}^{\pi}f(x)e^{-inx}dx \\
                    &=\frac{1}{2\pi}\int_{0}^{\pi}f(x)e^{-in(x-\pi)}dx+\frac{1}{2\pi}\int_{0}^{\pi}f(x)e^{-inx}dx \\
                    &=\frac{1}{2\pi}\int_{0}^{\pi}f(x)e^{-inx+in\pi}dx+\frac{1}{2\pi}\int_{0}^{\pi}f(x)e^{-inx}dx \\
                    &=\frac{1}{2\pi}\int_{0}^{\pi}f(x)e^{-inx}e^{in\pi}dx+\frac{1}{2\pi}\int_{0}^{\pi}f(x)e^{-inx}dx \\
                    &=\frac{1}{2\pi}\int_{0}^{\pi}f(x)e^{-inx}(-1)dx+\frac{1}{2\pi}\int_{0}^{\pi}f(x)e^{-inx}dx \\
                    &=0
                \end{aligned}
            \end{equation}
        \end{proof}
        \item[(5)] \begin{proof}[证明]
            \begin{equation}
                \begin{aligned}
                    f(x)
                    &\sim\hat{f}(0)+\sum_{n\geq1}(\hat{f}(n)+\hat{f}(-n))\cos nx+\sum_{n\geq1}i(\hat{f}(n)-\hat{f}(-n))\sin nx \\
                    &=Re(\hat{f}(0))+Im(\hat{f}(0)) \\
                    & \ \ \ \ +\sum_{n\geq1}(Re(\hat{f}(n))+Im(\hat{f}(n))+Re(\hat{f}(-n))+Im(\hat{f}(-n)))\cos nx \\
                    &\ \ \ \ +\sum_{n\geq1}i(Re(\hat{f}(n))+Im(\hat{f}(n))-Re(\hat{f}(-n))-Im(\hat{f}(-n)))\sin nx \\
                \end{aligned}
            \end{equation}
            由 $f$ 是实值函数得\begin{equation}
                \begin{aligned}
                    Im(\hat{f}(0))&=0 \\
                    Im(\hat{f}(n))&=-Im(\hat{f}(-n)) \\
                    Re(\hat{f}(n))&=Re(\hat{f}(-n))
                \end{aligned}
            \end{equation}
            进而得\begin{equation}
                \overline{\hat{f}(n)}=\hat{f}(-n)
            \end{equation}
            综上, 得证.
        \end{proof}
    \end{enumerate}
    \item \begin{proof}[证明]
    $\forall n\neq0, n\in\mathbb{Z}$
        \begin{equation}
            \begin{aligned}
                |\hat{f}(n)|
                &=\frac{1}{2\pi}\left|\int_{-\pi}^{\pi}f(x)e^{-inx}dx\right| \\
                &=\frac{1}{2\pi}\left|\frac{1}{-in}\int_{-\pi}^{\pi}f(x)de^{-inx}\right| \\
                &=\frac{1}{2\pi|n|}\left|i f(x)e^{-inx}|_{x=-\pi}^{\pi}-i\int_{-\pi}^{\pi}e^{-inx}df(x)\right| \\
                &=\frac{1}{2\pi|n|}\left|i\cos n\pi(f(\pi)-f(-\pi))-i\int_{-\pi}^{\pi}e^{-inx}f^{'}(x)dx\right| \\
                &=\frac{1}{2\pi|n|}\left|0-\frac{i}{-in}\int_{-\pi}^{\pi}f^{'}(x)de^{-inx}\right| \\
                &=\frac{1}{2\pi|n|^2}\left|\int_{-\pi}^{\pi}f^{'}(x)de^{-inx}\right| \\
                &=\frac{1}{2\pi|n|^2}\left|\cos n\pi(f^{'}(\pi)-f^{'}(-\pi))-\int_{-\pi}^{\pi}f^{''}(x)e^{-inx}dx\right| \\
                &=\frac{1}{2\pi|n|^2}\left|\int_{-\pi}^{\pi}f^{''}(x)e^{-inx}dx\right| \\
            \end{aligned}
        \end{equation}
        由于 $f$ 是 $C^2$ 光滑函数, 因此$\exists C\geq0$\begin{equation}
            \left|f^{''}(x)\right|\leq C
        \end{equation}
        进而\begin{equation}
            \begin{aligned}
                &\frac{1}{2\pi}\left|\int_{-\pi}^{\pi}f^{''}(x)e^{-inx}dx\right| \\
                &\leq\frac{1}{2\pi}\left|2\pi C\right| \\
                &=C
            \end{aligned}
        \end{equation}
        因此\begin{equation}
            |\hat{f}(n)|\leq\frac{C}{|n|^2}
        \end{equation}
        得证.
    \end{proof}
    \item \begin{proof}[证明]
        \begin{equation}
            \begin{aligned}
                f(x)
                &\sim\hat{f}(0)+\sum_{n\geq1}(\hat{f}(n)+\hat{f}(-n))\cos nx+i(\hat{f}(n)-\hat{f}(-n))\sin nx
            \end{aligned}
        \end{equation}
        由1.(3)知, 函数 $f$ 为正弦级数且 $\hat{f}(-n)=-\hat{f}(n)$, 因此:
        \begin{equation}
            f(x)\sim\sum_{n\geq1}2i\hat{f}(n)\sin nx
        \end{equation}
        由于 $f$ 为奇函数, 周期为 $2\pi$ 且在 $[0,\pi]$ 内, $f(x)=x(\pi-x)$, 可得:
        \begin{equation}
            \begin{aligned}
                \hat{f}(n)
                &=\frac{1}{2\pi}\int_{-\pi}^{\pi}f(x)e^{-inx}dx \\
                &=\frac{1}{2\pi}\int_{-\pi}^{0}x(x+\pi)e^{-inx}dx+\frac{1}{2\pi}\int_{0}^{\pi}x(\pi-x)e^{-inx}dx \\
                &=\frac{2i}{n^3\pi}(\cos n\pi-1) \\
            \end{aligned}
        \end{equation}
        因此:
        \begin{equation}
            \begin{aligned}
                f(x)
                &\sim\sum_{n\geq1}2i\hat{f}(n)\sin nx \\
                &=\sum_{n\geq1}2i\frac{2i}{n^3\pi}(\cos n\pi-1)\sin nx \\
                &=\sum_{n\geq1}\frac{4}{n^3\pi}(1-\cos n\pi)\sin nx \\
            \end{aligned}
        \end{equation}
    \end{proof}
    \item \begin{proof}[证明]
        由1.(4)知, 函数 $f$ 为余弦级数且 $\hat{f}(-n)=\hat{f}(n)$, 因此:
        \begin{equation}
            f(x)\sim\hat{f}(0)+\sum_{n\geq1}2\hat{f}(n)\cos nx
        \end{equation}
        由于 $f$ 为偶函数, 周期为 $2\pi$ 且在 $[-\pi,\pi]$ 内, $f(x)=\left|x\right|$, 可得:
        \begin{equation}
            \begin{aligned}
                \hat{f}(n)
                &=\frac{1}{2\pi}\int_{-\pi}^{\pi}f(x)e^{-inx}dx \\
                &=\frac{1}{2\pi}\int_{-\pi}^{\pi}|x|e^{-inx}dx \\
                &=\frac{1}{2\pi}\int_{-\pi}^{0}-xe^{-inx}dx+\frac{1}{2\pi}\int_{0}^{\pi}xe^{-inx}dx \\
                &=\frac{1}{2\pi n^2}(-ni\pi\cos n\pi-1+\cos n\pi)+\frac{1}{2\pi n^2}(ni\pi\cos n\pi+\cos n\pi-1) \\
                &=\frac{1}{\pi n^2}(\cos n\pi-1) \\
            \end{aligned}
        \end{equation}
        因此:
        \begin{equation}
            \begin{aligned}
                f(x)
                &\sim\hat{f}(0)+\sum_{n\geq1}2\hat{f}(n)\cos nx \\
                &=\frac{\pi}{2}+\sum_{n\geq1}\frac{2}{\pi n^2}(\cos n\pi-1)\cos nx \\
            \end{aligned}
        \end{equation}
    \end{proof}
    \item \begin{enumerate}
        \item[(1)] \begin{proof}[证明]
        $\forall x\in[-\pi,\pi]$
            \begin{equation}
                \left|\hat{f}(n)e^{inx}\right|\leq\left|\hat{f}(n)\right|
            \end{equation}
            并且已知
            \begin{equation}
                \sum_{n=-\infty}^{\infty}\left|\hat{f}(n)\right|
            \end{equation}
            收敛, 因此由 Weierstrass 判别法知
            \begin{equation}
                \sum_{n=-\infty}^{\infty}\hat{f}(n)e^{inx}
            \end{equation}
            在 $[-\pi,\pi]$ 上一致收敛.
        \end{proof}
        \item[(2)] \begin{proof}[解]
            \begin{equation}
                g(x)=\sum_{n=-\infty}^{\infty}\hat{f}(n)e^{inx}
            \end{equation}
            可得:
            \begin{equation}
                \begin{aligned}
                    \hat{g}(n)
                    &=\frac{1}{2\pi}\int_{-\pi}^{\pi}g(x)e^{-inx}dx \\
                    &=\frac{1}{2\pi}\int_{-\pi}^{\pi}(\sum_{k=-\infty}^{\infty}\hat{f}(k)e^{ikx})e^{-inx}dx \\
                    &=\frac{1}{2\pi}\sum_{k=-\infty}^{\infty}(\hat{f}(k)\int_{-\pi}^{\pi}e^{i(k-n)x}dx) \\
                    &=\frac{1}{2\pi}\sum_{k=-\infty}^{n-1}(\hat{f}(k)\times0)+\frac{1}{2\pi}\hat{f}(n)\times2\pi+\frac{1}{2\pi}\sum_{n+1}^{\infty}(\hat{f}(k)\times0) \\
                    &=\hat{f}(n)
                \end{aligned}
            \end{equation}
        \end{proof}
        \item[(3)] \begin{proof}[证明]
            令 $h=f-g$ , 则\begin{equation}
                \hat{h}(n)=\hat{f}(n)-\hat{g}(n)=0
            \end{equation}
            由定理 2.1 知, $h=0$ , 则$f=g$ .
        \end{proof}
        \item[(4)] \begin{proof}[证明]
            \begin{equation}
                \begin{aligned}
                    f(0)=\frac{\pi^2}{4}
                \end{aligned}
            \end{equation}
            令\begin{equation}
                g(x)=\frac{\pi^2}{12}+\sum_{n=1}^{\infty}\frac{\cos nx}{n^2}
            \end{equation}
            则
            \begin{equation}
                \begin{aligned}
                    g(0)
                    &=\frac{\pi^2}{12}+\sum_{n=1}^{\infty}\frac{\cos nx}{n^2}\bigg|_{x=0} \\
                    &=\frac{\pi^2}{12}+\sum_{n=1}^{\infty}\frac{1}{n^2}
                \end{aligned}
            \end{equation}
            由(3)知, $f=g$ , 因此得\begin{equation}
                \begin{aligned}
                    \sum_{n=1}^{\infty}\frac{1}{n^2}
                    &=\frac{\pi^2}{4}-\frac{\pi^2}{12} \\
                    &=\frac{\pi^2}{6}
                \end{aligned}
            \end{equation}
        \end{proof}
    \end{enumerate}
\end{enumerate}
\end{document}