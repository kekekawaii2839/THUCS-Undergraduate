\documentclass{article}
\usepackage{amsmath}
\usepackage{amsthm}
\usepackage{amssymb}
\usepackage[a4paper, left=1.5cm, right=1.5cm, top=2.5cm, bottom=2.5cm]{geometry}
\usepackage{tikz}
\usetikzlibrary{tikzmark}
\usepackage{stata}

\title{PS3}

\begin{document}
\maketitle
\begin{enumerate}
    \item \begin{enumerate}
        \item \begin{equation}
            \begin{aligned}
                H&=\sum_{i=1}^{n}(y_i-\hat{\beta}_0-\hat{\beta}_1d_i-\hat{\beta}_2z_i-\hat{\beta}_3d_iz_i)^2 \\
                \frac{\partial H}{\partial \hat{\beta}_0}&=2\sum_{i=1}^{n}(y_i-\hat{\beta}_0-\hat{\beta}_1d_i-\hat{\beta}_2z_i-\hat{\beta}_3d_iz_i)(-1)=0 \\
                \frac{\partial H}{\partial \hat{\beta}_1}&=2\sum_{i=1}^{n}(y_i-\hat{\beta}_0-\hat{\beta}_1d_i-\hat{\beta}_2z_i-\hat{\beta}_3d_iz_i)(-d_i)=0 \\
                \frac{\partial H}{\partial \hat{\beta}_2}&=2\sum_{i=1}^{n}(y_i-\hat{\beta}_0-\hat{\beta}_1d_i-\hat{\beta}_2z_i-\hat{\beta}_3d_iz_i)(-z_i)=0 \\
                \frac{\partial H}{\partial \hat{\beta}_3}&=2\sum_{i=1}^{n}(y_i-\hat{\beta}_0-\hat{\beta}_1d_i-\hat{\beta}_2z_i-\hat{\beta}_3d_iz_i)(-d_iz_i)=0 \\
            \end{aligned}
            \nonumber
        \end{equation}
        \item $\bar{y}_{11}$ represents the average income of rural female, 
        $\bar{y}_{10}$ represents the average income of urban female,
        $\bar{y}_{01}$ represents the average income of rural male and
        $\bar{y}_{00}$ represents the average income of urban male.
        \item \begin{equation}
            \begin{aligned}
                &\quad\sum_{i=1}^{n}d_iz_i(y_i-b_0-b_1d_i-b_2z_i-b_3d_iz_i) \\
                &=\sum_{i=1}^{n}d_iz_i\left[y_i+\bar{y}_{00}(-1+d_i+z_i-d_iz_i)+\bar{y}_{01}(-z_i+d_iz_i)+\bar{y}_{10}(-d_i+d_iz_i)+\bar{y}_{11}(-d_iz_i)\right] \\
                &=\sum_{i=1}^{n}\left[y_id_iz_i+\bar{y}_{00}(-d_iz_i+d_iz_i+d_iz_i-d_iz_i)+\bar{y}_{01}(-d_iz_i+d_iz_i)+\bar{y}_{10}(-d_iz_i+d_iz_i)+\bar{y}_{11}(-d_iz_i)\right] \\
                &=\sum_{i=1}^{n}\left[y_id_iz_i+\bar{y}_{00}\cdot0+\bar{y}_{01}\cdot0+\bar{y}_{10}\cdot0+\bar{y}_{11}(-d_iz_i)\right] \\
                &=\sum_{i=1}^{n}[y_id_iz_i+\bar{y}_{11}(-d_iz_i)] \\
                &=\sum_{i=1}^{n}y_id_iz_i+\bar{y}_{11}\sum_{i=1}^{n}(-d_iz_i) \\
                &=\sum_{i=1}^{n}y_id_iz_i-\frac{\sum_{i=1}^{n}(y_id_iz_i)}{\sum_{i=1}^{n}(d_iz_i)}\sum_{i=1}^{n}(d_iz_i) \\
                &=0 \\
            \end{aligned}
            \nonumber
        \end{equation}
        \item \begin{equation}
            \begin{aligned}
                \mathbb{E}(\bar{y}_{11})
                &=\mathbb{E}\left(\frac{\sum_{i=1}^{n}y_id_iz_i}{\sum_{i=1}^{n}d_iz_i}\right) \\
                &=\frac{\sum_{i=1}^{n}d_iz_i\mathbb{E}(y_i)}{\sum_{i=1}^{n}d_iz_i} \\
                &=\mathbb{E}(y|d=1,z=1) \\
            \end{aligned}
            \nonumber
        \end{equation} \par
        Similarly, we can get $\mathbb{E}(\bar{y}_{10})=\mathbb{E}(y|d=1,z=0)$, $\mathbb{E}(\bar{y}_{01})=\mathbb{E}(y|d=0,z=1)$ and $\mathbb{E}(\bar{y}_{00})=\mathbb{E}(y|d=0,z=0)$.
        Then consider the following equations:
        \begin{equation}
            \begin{aligned}
                \beta_0&=\mathbb{E}(\hat{\beta}_0)=\mathbb{E}(b_0)=\mathbb{E}(\bar{y}_{00})=\mathbb{E}(y|d=0,z=0) \\
                \beta_1&=\mathbb{E}(\hat{\beta}_1)=\mathbb{E}(b_1)=\mathbb{E}(\bar{y}_{01})-\mathbb{E}(\bar{y}_{00})=\mathbb{E}(y|d=0,z=1)-\mathbb{E}(y|d=0,z=0) \\
                \beta_2&=\mathbb{E}(\hat{\beta}_2)=\mathbb{E}(b_2)=\mathbb{E}(\bar{y}_{10})-\mathbb{E}(\bar{y}_{00})=\mathbb{E}(y|d=1,z=0)-\mathbb{E}(y|d=0,z=0) \\
                \beta_3&=\mathbb{E}(\hat{\beta}_3)=\mathbb{E}(b_3)=\mathbb{E}(\bar{y}_{11}-\bar{y}_{10}-\bar{y}_{01}+\bar{y}_{00}) \\
                &=\mathbb{E}(y|d=1,z=1)-\mathbb{E}(y|d=1,z=0)-\mathbb{E}(y|d=0,z=1)+\mathbb{E}(y|d=0,z=0) \\
            \end{aligned}
            \nonumber
        \end{equation}
        \item The disparity in income improvement effects between men and women transitioning from urban to rural areas.
        \item $H_0: \beta_1+\beta_3=0$ , $H_1: \beta_1+\beta_3\neq0$
        Then we can use F-test to test the hypothesis.
        The restrcited model is $y_i=\beta_0+\beta_1d_i+\beta_2z_i-\beta_1d_iz_i+u_i$ .
        Calulate the F-statistic $F=\frac{(R_{ur}^2-R_r^2)/1}{R_r^2/(n-4)}$ and compare it with the critical value. 
        If $F>F_{1-0.025,1,n-4}$, we reject the null hypothesis.
        \item $H_0: \beta_1=\beta_3=0$ . 
        The restrcited model is $y_i=\beta_0+\beta_2z_i+u_i$ .
        The steps are similar to the previous question, but change the first parameter of F-statistic to 2,
        which is $F=\frac{(R_{ur}^2-R_r^2)/2}{R_r^2/(n-4)}$ . And compare it with the critical value $F_{1-0.025,2,n-4}$.
        If the F-statistic is larger, we reject the null hypothesis.
    \end{enumerate}
    \item \begin{enumerate}
        \item \begin{equation}
            \begin{aligned}
                \mathbb{E}(\hat{\beta}_1)-\tau_{ATE}
                &=\mathbb{E}\left[\frac{1}{n_1}\sum_{i=1}^{n}d_iy_i-\frac{1}{n_0}\sum_{i=1}^{n}(1-d_i)y_i\right]-\mathbb{E}(y(1)-y(0)) \\
                &=\frac{\sum_{i=1}^{n}d_i\mathbb{E}(y_i)}{\sum_{i=1}^{n}d_i}-\frac{\sum_{i=1}^{n}(1-d_i)\mathbb{E}(y_i)}{\sum_{i=1}^{n}(1-d_i)}-\mathbb{E}(y(1))+\mathbb{E}(y(0)) \\
                &=\mathbb{E}(y(1)|d=1)-\mathbb{E}(y(0)|d=0)-\mathbb{E}(y(1))+\mathbb{E}(y(0)) \\
                &=\mathbb{E}(y(1)|d=1)-\mathbb{E}(y(0)|d=0)-[\mathbb{E}(y(1)|d=1)\cdot p_1+\mathbb{E}(y(1)|d=0)\cdot (1-p_1)] \\
                &\quad+[\mathbb{E}(y(0)|d=1)\cdot p_1+\mathbb{E}(y(0)|d=0)\cdot (1-p_1)] \\
                &=\left(\mathbb{E}(y(1)|d=1)-\mathbb{E}(y(1)|d=0)\right)(1-p_1)+\left(\mathbb{E}(y(0)|d=1)-\mathbb{E}(y(0)|d=0)\right)p_1 \\
            \end{aligned}
            \nonumber
        \end{equation}
        \item Since we have $$
            y=\beta_0+\beta_1d+u=dy(1)+(1-d)y(0)
        $$ , so we can get $$
            y(1)=\beta_1+\beta_0+u \quad y(0)=\beta_0+u
        $$ , which implys that $$
            \mathbb{E}(y(1)|d=1)=\mathbb{E}(y(1)|d=0) \quad \mathbb{E}(y(0)|d=1)=\mathbb{E}(y(0)|d=0)
        $$ , then we can get $$
            \mathbb{E}(\hat{\beta}_1)-\tau_{ATE}=0
        $$ .
        If we further assume that $\mathbb{E}[u|d]=0$, it won't affect the result.
        \item \begin{equation}
            \begin{aligned}
                y
                &=\beta_0^{'}+\tau_{ATE}d+u^{'} \\
                &=\beta_0^{'}+\mathbb{E}[y(1)-y(0)]d+u^{'} \\
                &=\beta_0^{'}+[(\beta_1+\beta_0+u)-(\beta_0+u)]d+u^{'} \\
                &=\beta_0^{'}+\beta_1d+u^{'} \\
                &=\beta_0+\beta_1d+u \\
            \end{aligned}
            \nonumber
        \end{equation} par
        So, we can get $\beta_0^{'}=\beta_0$ and $u^{'}=u$.
    \end{enumerate}
    \item \begin{enumerate}
        \item No, since \textit{attend} can be affected by \textit{alcohol} (like students who drink too much will not attend the class tomorrow) .
        And we only want to estimate the effect of \textit{alcohol} on \textit{colGPA} .
        If we add \textit{attend} into the model, the coefficient of \textit{alcohol} 's meaning will be change to the effect of \textit{alcohol} on \textit{colGPA} when \textit{attend} is fixed, the power of \textit{alcohol} will be weakened.
        \item No, \textit{gaokaoScore} and \textit{hsGPA} can both be affected by \textit{alcohol}, they'll weaken the power of \textit{alcohol} on \textit{colGPA} if we add them into the model, just like the previous question.
    \end{enumerate}
    \item \begin{enumerate}
        \item See the log file.
        \item $\hat{\beta}_1$ means holding \textit{restaurn}, people between 30 and 50 years old have 3.1 more cigarettes smoked per day than people below 30 years old. \\
        $\hat{\beta}_1$ means holding \textit{restaurn}, people between 50 and 70 years old have 0.92 more cigarettes smoked per day than people below 30 years old. \\
        $\hat{\beta}_1$ means holding \textit{restaurn}, people above 70 years old have 5.8 less cigarettes smoked per day than people below 30 years old. \\
        \item The marginal effect of \textit{age} on \textit{cigs} is $\frac{\partial \textit{cigs}}{\partial \textit{age}}=\theta_1+2\theta_2\textit{age}$ .
        Solve: $\hat{\theta}_1+2\hat{\theta}_2\cdot\textit{age}=0$, we get $\textit{age}=43$ .
        \item Holding other factors fixed, the decrease in smoking per day if the city requires no smoking in restaurants.
        \item \begin{equation}
            \frac{\partial E(\textit{cigs})}{\partial\textit{educ}}=\begin{cases}
                \gamma_1 & \text{if }\textit{restaurn}=0 \\
                \gamma_1+\gamma_3 & \text{if }\textit{restaurn}=1 \\
            \end{cases}
        \end{equation} \par
        $\gamma_3$ means the increase of the marginal effect of \textit{educ} on smoking per day if the city requires no smoking in restaurants.
        \item From the Stata output, we know the p-value for $\gamma_3=0$ is 0.890, which is smaller than 0.95, so we can reject the null hypothesis. \par
        $\gamma_3$ is significant at the 5\% level.
    \end{enumerate}
\end{enumerate}
\end{document}