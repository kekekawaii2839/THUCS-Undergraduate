\documentclass{article}
\usepackage{amsmath}
\usepackage{amsthm}
\usepackage{amssymb}
\usepackage[a4paper, left=1.5cm, right=1.5cm, top=2.5cm, bottom=2.5cm]{geometry}
\usepackage{tikz}
\usetikzlibrary{tikzmark}

\title{HW2}
\author{Li Haolun\ 2022011545}

\begin{document}
\maketitle
\begin{enumerate}
    \item \begin{enumerate}
        \item[(a)] 
        \begin{equation}
            \begin{aligned}
                k(x_1,x_2)
                &=-\frac{\frac{\partial U}{\partial x_1}}{\frac{\partial U}{\partial x_2}} \\
                &=-\frac{3x_1^2x_2^5}{5x_1^3x_2^4} \\
                &=-\frac{3x_2}{5x_1}
            \end{aligned}
            \nonumber
        \end{equation}
        \item[(b)] We have $\frac{3x_2}{5x_1}=\frac{p_1}{p_2}$ , which is eqivalent to $5p_1x_1=3p_2x_2$. 
        So the expenditure share of goods 1 is $\frac{3}{3+5}=\frac{3}{8}$ .
        \item[(c)] 
        \begin{equation}
            \begin{aligned}
                k(x_1,x_2)
                &=-\frac{\frac{\partial U}{\partial x_1}}{\frac{\partial U}{\partial x_2}} \\
                &=-\frac{ax_2}{bx_1} \\
                &=-\frac{p_1}{p_2}
            \end{aligned}
            \nonumber
        \end{equation} \par
        $\Rightarrow bp_1x_1=ap_2x_2$ , so the expenditure share of goods 1 is $\frac{a}{a+b}$ .
    \end{enumerate}
    \item \begin{enumerate}
        \item[(a)] 
        \begin{equation}
            \begin{aligned}
                k(x_1,x_2)
                &=-\frac{\frac{\partial U}{\partial x_1}}{\frac{\partial U}{\partial x_2}} \\
                &=\frac{1}{100-2x_2} \\
                &=-\frac{p_1}{p_2} \\
                &=-\frac{1}{4}
            \end{aligned}
            \nonumber
        \end{equation} \par
        $\Rightarrow$ $4=100-2x_2$ \par
        $\Rightarrow$ $x_1^*=308$ , $x_2^*=48$ .
        \item[(b)] We still have $4=100-2x_2$ , so $x_1^*=808$ , $x_2^*=48$ .
        \item[(c)] From the utility function, we always have $4=100-2x_2$ , which means that $x_2$ is always $48$ . \par
        So, if the consumer consumes both goods, her income $a$ must be larger than $48\times4=192$ . \par
    \end{enumerate}
    \item \begin{enumerate}
        \item[(a)] $q=0.02\times7500-3\times30=60$ .
        \item[(b)] Before the price of a blind box rose, Li Hua spent $5700$ Yuan on all other goods.
        So, Li Hua need $5700+60\times40=8100$ Yuan after the price rose. \par
        And he would buy $q^{'}=0.02\times8100-3\times40=42$ boxes at this income.
        \item[(c)] $q^{''}=0.02\times7500-3\times40=30$ .
        \item[(d)] Substitution effect: $q^{'}-q=-18$ . \par
        Income effect: $q^{''}-q^{'}=-12$ .
    \end{enumerate}
    \item \begin{enumerate}
        \item[(a)] \leavevmode\vadjust{\vspace{-\baselineskip}}\newline
        \begin{figure}[!htb]
            \centering
            \begin{tikzpicture}
                \draw[->] (0,0) -- (6,0) node[right] {$h$};
                \draw[->] (0,0) -- (0,6) node[above] {$s$};
                \draw[domain=0:5, samples=100] plot (\x, {2.5-0.5*\x});
                \node[right] at(4,1) {$1000h+2000s=200000$};
                \draw[domain=0:15/4, samples=100] plot (\x, {15/4-\x});
                \node[right] at(1,3) {$h+s=150$};
                \fill (2.5,5/4) circle (0.05);
                \node[above] at(3,5/4) {(100,50)};
            \end{tikzpicture}
        \end{figure} \par
        $1000h+2000s=200000$ means the hotel manager's budget constraint. \par
        $h+s=150$ means the hotel manager's reality constraint. \par
        So, he will buy $100$ hard mattresses. \par
        \item[(b)] \leavevmode\vadjust{\vspace{-\baselineskip}}\newline
        \begin{figure}[!htb]
            \centering
            \begin{tikzpicture}
                \draw[->] (0,0) -- (6,0) node[right] {$h$};
                \draw[->] (0,0) -- (0,6) node[above] {$s$};
                \draw[domain=0:5, samples=100] plot (\x, {2.5-0.25*\x});
                \node[right] at(4,1) {$500h+2000s=200000$};
                \draw[domain=0:15/4, samples=100] plot (\x, {15/4-\x});
                \node[right] at(0.5,3.5) {$h+s=150$};
                \fill (5/3,25/12) circle (0.05);
                \node[above] at(3,2) {($\frac{200}{3}$ , $\frac{250}{3}$)};
            \end{tikzpicture}
        \end{figure} \par
        Now he will buy $\frac{200}{3}$ hard mattresses and $\frac{250}{3}$ soft mattresses. \par
        Hard mattress is Giffen good for him.
    \end{enumerate}
    \item \begin{enumerate}
        \item[(a)] 
        \begin{equation}
            \begin{aligned}
                k(c,r)
                &=-\frac{\frac{\partial U}{\partial r}}{\frac{\partial U}{\partial c}} \\
                &=-\frac{c}{r} \\
            \end{aligned}
            \nonumber
        \end{equation} \par
        For Magneto, her budget constraint is $c+50r=4000$ . \par
        So we have $\frac{c}{r}=50$ . And the solution is $c=2000$ , $r=40$ .
        \item[(b)] \begin{equation}
            \begin{cases}
                c+40r=3200 & r\geq40 \\
                c+60r=4000 & r<40
            \end{cases}
            \nonumber
        \end{equation}
        \item[(c)] No, he won't. \par
        When $r\geq40$ , we have $\frac{c}{r}=40$ , then $r=40$ . \par
        When $r<40$ , we have $\frac{c}{r}=60$ , then $r=\frac{100}{3}$ . \par
        The best choice for Xavier is $c=2000$ , $r=\frac{100}{3}$ .
        \item[(d)] Magneto has the better job, because $U_M=2000\times\frac{100}{3}>U_X=1600\times40$ .
    \end{enumerate}
    \item \begin{enumerate}
        \item[(a)] He will choose 10 hours for leisure.
        \item[(b)] His budget constraint is $C+10R=190$ . \par
        \begin{equation}
            \begin{aligned}
                k(C,R)
                &=-\frac{\frac{\partial U}{\partial R}}{\frac{\partial U}{\partial C}} \\
                &=2R-20 \\
                &=-10
            \end{aligned}
            \nonumber
        \end{equation} \par
        So we have $R=5$ , $C=140$ . He will choose to work 11 hours.
        \item[(c)] Now his budget constraint is $C+8R=190\times0.8=152$ . \par
        We still have $R=6$ , then $C=104$ . He will choose to work 10 hours.
    \end{enumerate}
    \item \begin{enumerate}
        \item[(a)] $m_1+\frac{m_2}{1+r}=c_1+\frac{c_2}{1+r}$ .
        \item[(b)] He can consume $m_1(1+r)+m_2$ , which is the future value of his endowment.
        \item[(c)] He can consume $m_1+\frac{m_2}{(1+r)}$ , which is the present value of his endowment.
        \item[(d)] Slope is $-(1+r)$ .
    \end{enumerate}
    \item \begin{enumerate}
        \item[(a)] In this question, $m_1=1000$ , $m_2=150$ , $r=-0.25$ . \par
        So villagers' budget constraint in present value terms is $1200=c_1+\frac{4}{3}c_2$ . \par
        \begin{equation}
            \begin{aligned}
                k(c_1,c_2)
                &=-\frac{\frac{\partial U}{\partial c_1}}{\frac{\partial U}{\partial c_2}} \\
                &=-\frac{c_2}{c_1} \\
                &=-(1+r)=-0.75
            \end{aligned}
            \nonumber
        \end{equation} \par
        $\Rightarrow$ $c_1=600$ , $c_2=450$ .
        \item[(b)] Since the villagers can sell and buy wheat in any year, 
        and the price is \$1 per kilo, so villagers' money can be seen as wheat, and they don't have to suffer from the loss during storage. \par
        So villagers' budget constraint in present value terms is $1000+150/(1+0.1)=\frac{12500}{11}=c_1+\frac{c_2}{1.1}$ . \par
        We have $\frac{c_2}{c_1}=1.1$ , then $c_2=625$  \par
        Now, the villagers will consume $c_1=\frac{6250}{11}$ , $c_2=625$ .
    \end{enumerate}
    \item \begin{enumerate}
        \item[(a)] $P(m)=20=40-2m\Rightarrow m=10$ . His consumer's surplus is $\frac{20\times10}{2}=100$ . 
        \item[(b)] Now he will consume $15$ cups per month, and his consumer's surplus is $\frac{30\times15}{2}=225$ . \par
        So Mr. Hacker’s change in consumer’s surplus is $+125$ .
    \end{enumerate}
    \item \begin{enumerate}
        \item[(a)] Oliver has quasilinear utility function. His inverse demand function is $P(x)=60-x$ .
        \item[(b)] He will consume $30$ bottles of wine.
        \item[(c)] $10$ .
        \item[(d)] In (b), his consumer's surplus is $\frac{30\times30}{2}=450$ . \par
        In (c), his consumer's surplus is $\frac{10\times10}{2}=50$ . \par
        So, the change in consumer’s surplus is $-400$ .
    \end{enumerate}
\end{enumerate}
\end{document}