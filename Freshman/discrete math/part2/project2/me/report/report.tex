\documentclass{article}
\usepackage{ctex}
\usepackage{amssymb}
\usepackage{minted}
\title{图论project报告}
\date{2023-06-02}
\begin{document}
\maketitle
\section{半监督的图节点分类问题的建模过程}
对于半监督的图节点分类的问题, 仅有一小部分的节点存在有效的标签. 
因此我们可以将整个图结构直接用神经网络模型的方式编码, 然后在这些存在标签的节点上进行训练.

对于图的数学表达, 我们可以选用邻接矩阵的方式. 对于每个节点的标签, 实际上是每个节点都有一个与之对应的特征向量.
我们就可以用图卷积网络(GCN)对图进行卷积操作, 使得节点的特征向量同时包含了自身和邻居节点的特征信息.
由于是半监督的问题, 我们还需要对节点的标签进行传播, 而多层的GCN就可以实现这一点.
最后我们可以定义好激活函数与损失函数, 用以度量预测标签与真实标签的差距, 就可以训练模型了.
\section{图卷积网络算法的具体过程}
在论文中, 作者给出了图卷积网络的数学推导, 最终选择了两层线性GCN层组成的结构. 以下是最终推出的式子:
$$Z=f(X,A)=softmax(\mathop{A}\limits^{\wedge}ReLU(\mathop{A}\limits^{\wedge}XW^{(0)})W^{(1)})$$
其中$X\in \mathbb{R}^{N \times C}$表示每个节点的$C$维特征向量组成的矩阵, $A$是邻接矩阵, $W^{(0)}\in\mathbb{R}^{C \times H}$是输入层到隐藏层的参数矩阵, $W^{(1)}\in\mathbb{R}^{H \times F}$是隐藏层到输出层的参数矩阵, $H$为隐藏层的维数.
定义$\mathop{A}\limits^{\wedge}={\mathop{D}\limits^{\sim}}^{-\frac{1}{2}}\mathop{A}\limits^{\sim}{\mathop{D}\limits^{\sim}}^{-\frac{1}{2}}$,
其中$\mathop{A}\limits^{\sim}=A+I_N$, $I_N$是$N$维单位矩阵, $\mathop{D}\limits^{\sim}$是$\mathop{A}\limits^{\sim}$的度矩阵.

在GCN中, 每个节点的特征向量都包含了自身的特征, 每个节点都与自身相连, 因此在邻接矩阵中加上单位矩阵, 可以使得每个节点的特征向量都包含了自身的特征.
而与度矩阵计算, 可以使得每个节点的特征向量都包含了邻居节点的特征, 从而实现了卷积操作.
这里的第一层GCN选用了ReLU作为激活函数, 第二层GCN选用了softmax作为激活函数, 从而实现了标签的传播.

在实际训练中, 将使用梯度下降法作为对参数矩阵$W$的训练方法. 而损失函数则选用了交叉熵损失函数.

我复现论文中的实验, 使用了python3.7以及tensorflow 1.15.4版本.
上述的激活函数与损失函数, tensorflow都有现成的实现, 直接调用即可.

代码思路主要为通过继承tf.keras.Model和tf.keras.layers.Layer类, 重写其中的call方法, 从而实现自定义的GCN.
而论文中对邻接矩阵的归一化处理, 我将其放在了数据读取的部分, 随其他相关代码一起封装在了DataSets类中.
具体代码见附件.
\section{实验结果}
我使用了Cora和Citeseer两个数据集进行实验, 超参数均与论文保持一致(即隐藏层维数为16, 学习率为0.01, dropout为0.5, L2正则化系数为$5\times 10^{-4}$).
而标签率也与论文基本保持一致, 即Cora数据集的标签率为4\%, Citeseer数据集的标签率为5\%(清洗后).
以下实验数据是随机测试20次后的平均结果:
\begin{table}[htbp]
    \centering
    \caption{实验结果}
    \begin{tabular}{|c|c|c|c|}
        \hline
        \textbf{损失函数} & \textbf{标签率} & \textbf{准确率} & \textbf{偏差范围} \\
        \hline
        \textbf{Cora} & 4\% & 0.76 & $\pm$ 0.02 \\
        \textbf{Citeseer} & 5\% & 0.56 & $\pm$ 0.01 \\
        \hline
    \end{tabular}
\end{table}

特别感谢Github Copilot以及ChatGPT的帮助, 让我能快速上手tensorflow.
\end{document}