\documentclass{exam}
\usepackage[heading=true]{ctex}
\usepackage{amsmath}
\usepackage{amssymb}
\usepackage{tikz}
\title{Calculus T(1) Final}
\date{December 30, 2022}
\begin{document}
\maketitle
\section{填空题}
\begin{questions}
    \question[3]设$f(x)$在$[0,+\infty)$上连续, 且$\int_{0}^{x^2}f(t)dt=2x^5,\forall x \in (0,+\infty)$, 则$f(1)=\underline{\hspace{1.5cm}}$.
    \question[3]设$y(x)$是一阶线性方程$y^{'}-\frac{y}{x}=x (x>0)$满足初值条件$y(1)=1$的解, 则$y(2)=\underline{\hspace{1.5cm}}$.
    \question[3]已知曲线$y=y(x)$经过点$(-1,1)$, 且该曲线上任意点$P$处切线的斜率是直线$OP$的三倍, 其中$O$是原点, 则$y(-2)=\underline{\hspace{1.5cm}}$.
    \question[3]积分$\int_{0}^{\frac{\pi}{2}}3\sqrt{sin x-sin^3x}dx=\underline{\hspace{1.5cm}}$.
    \question[3]设连续可微函数$y=y(x)$由方程$x=\int_{1}^{y}sin^2(\frac{\pi t}{4})dt$确定, 则$y^{'}(0)=\underline{\hspace{1.5cm}}$.
    \question[3]曲线段$y=\int_{0}^{x}\sqrt{sin t}dt (0 \leq x \leq \pi)$的弧长为$\underline{\hspace{1.5cm}}$.
    \question[3]曲线$y=\frac{1}{x}+ln(1+e^x)$的渐近线共有$\underline{\hspace{1.5cm}}$条.
    \question[3]设$y(x)$是常微分方程$yy^{''}+(y^{'})^2=1$满足初值条件$y(0)=1,y^{'}(0)=0$的解, 则$(y(1))^2=\underline{\hspace{1.5cm}}$.
    \question[3]积分$\int_{-1}^{1}\frac{sin x dx}{\sqrt[3]{1-x}+\sqrt[3]{1+x}}=\underline{\hspace{1.5cm}}$.
    \question[3]已知$y(x)=xe^{-2x}$是常系数二阶线性齐次方程$y^{''}+ay^{'}+by=0$的解, 则$a+b=\underline{\hspace{1.5cm}}$.
\end{questions}
\section{选择题}
\begin{questions}
    \question[3]极限$\lim\limits_{n \to +\infty}\sum_{k=1}^{n}(1+\frac{k}{n})\frac{k}{n^2}=$
    \begin{choices}
        \choice $\frac{2}{3}$;
        \choice $\frac{1}{2}$;
        \choice $\frac{5}{6}$;
        \choice $\frac{1}{3}$.
    \end{choices}
    \question[3]积分$\int_{-\sqrt{2}}^{\sqrt{2}}\frac{dx}{x^2\sqrt{x^2-1}}$等于
    \begin{choices}
        \choice $\frac{1}{2}(\sqrt{6}-2)$;
        \choice $\frac{1}{2}(\sqrt{3}-\sqrt{2})$;
        \choice $\frac{1}{2}$;
        \choice $\frac{1}{2}(2-\sqrt{3})$.
    \end{choices}
    \question[3]设二阶线性常系数常微分方程的通解为$y=e^x(C_1cosx+C_2sinx)+x$, 其中$C_1$和$C_2$为任意常数, 则这个微分方程是
    \begin{choices}
        \choice $y^{''}-2y^{'}+2y=2x-2$;
        \choice $y^{''}+2y^{'}+2y=2x-2$;
        \choice $y^{''}-2y^{'}+2y=2x+2$;
        \choice $y^{''}+2y^{'}+2y=2x+2$.
    \end{choices}
    \question[3]函数$e^{sinx}$在一个最小正周期内的拐点个数为
    \begin{choices}
        \choice 2;
        \choice 4;
        \choice 1;
        \choice 0.
    \end{choices}
    \question[3]设$\lambda(x)$是周期为$2$的分段常值函数, 其图像如下.\\
    \begin{tikzpicture}
        \draw[->](-3.2,0)--(3.2,0) node[right]{$X$};
        \draw[->](0,-3.2)--(0,3.2) node[right]{$Y$};
        \draw(-3,0)--(-3,0.1) node[below=3]{$-3$};
        \draw(-2,0)--(-2,0.1) node[below=3]{$-2$};
        \draw(-1,0)--(-1,0.1) node[below=3]{$-1$};
        \draw(0,0)--(0,0.1) node[below=3]{$O$};
        \draw(1,0)--(1,0.1) node[below=3]{$1$};
        \draw(2,0)--(2,0.1) node[below=3]{$2$};
        \draw(3,0)--(3,0.1) node[below=3]{$3$};
        \draw(0,1)--(0.1,1) node[left=3]{$1$};
        \draw(0,-1)--(0.1,-1) node[right]{$-1$};
        \draw[thick,blue] (-1,-1) -- (0,-1);
        \draw[thick,blue] (0,1) -- (1,1);
        \draw[thick,blue] (-3,-1) -- (-2,-1);
        \draw[thick,blue] (2,1) -- (3,1);
        \draw[thick,blue] (1,-1) -- (2,-1);
        \draw[thick,blue] (-2,1) -- (-1,1);
        \draw[dashed,red] (1,1) -- (1,-1);
        \draw[dashed,red] (2,1) -- (2,-1);
        \draw[dashed,red] (-1,1) -- (-1,-1);
        \draw[dashed,red] (-2,1) -- (-2,-1);
    \end{tikzpicture}
    \begin{choices}
        \choice 微分方程$y^{'}+\lambda(x)y=0$的所有解都是周期函数;
        \choice 微分方程$y^{'}+\lambda(x)y=0$既有无穷多个不恒等于零的周期解, 同时也有无穷多个非周期解;
        \choice 微分方程$y^{'}+\lambda(x)y=0$具有有限多个的非周期解;
        \choice 除零解外, 微分方程$y^{'}+\lambda(x)y=0$没有周期解.
    \end{choices}
    \question[3]曲线段$y=\sqrt{x}(0\leq x\leq 2)$绕$x$轴旋转一周所得旋转面面积为
    \begin{choices}
        \choice $\frac{13\pi}{3}$;
        \choice $\frac{8\sqrt{2}\pi}{3}$;
        \choice $2\pi$;
        \choice $\frac{4(3\sqrt{3}-1)\pi}{3}$.
    \end{choices}
    \question[3]设$f(x)$为定义在闭区间$[a,b]$上的函数. 下列命题中的错误命题是
    \begin{choices}
        \choice 若$f(x)$在$[a,b]$上单调, 则$f(x)$在$[a,b]$上可积;
        \choice 若$f(x)$在$[a,b]$上非负可积且$\int_{a}^{b}f(x)dx=0$, 则$f(x)$在其连续点处为零;
        \choice 若$f(x)$在$[a,b]$上有界, 则$f(x)$在$[a,b]$上可积;
        \choice 若$f(x)$在$[a,b]$上有界, 则$f(x)$在$[a,b]$上的(达布)上下积分均存在.
    \end{choices}
    \question[3]定义$J_k=\int_{0}^{k\pi}e^{x^2}sinxdx$, $k=1,2,3$. 则这三个积分值从小到大排列依次为
    \begin{choices}
        \choice $J_2$, $J_3$, $J_1$;
        \choice $J_1$, $J_2$, $J_3$;
        \choice $J_2$, $J_1$, $J_3$;
        \choice $J_3$, $J_2$, $J_1$.
    \end{choices}
    \question[3]设$f(x)=
    \begin{cases}
        \frac{2x+1}{x^2+x+1},& x \geq 0,\\
        2x+1,& x < 0.
    \end{cases}$若$F(x)$为$f(x)$的一个原函数, 且$F(-1)=1$, 则
    \begin{choices}
        \choice $F(x)=
        \begin{cases}
            ln(x^2+x+1),& x \geq 0,\\
            x^2+x-1,& x < 0.
        \end{cases}$
        \choice $F(x)=
        \begin{cases}
            ln(x^2-x+1)+1,& x \geq 0,\\
            x^2+x+1,& x < 0.
        \end{cases}$
        \choice $F(x)=
        \begin{cases}
            ln(x^2+x+1)+1,& x \geq 0,\\
            x^2+x+1,& x < 0.
        \end{cases}$
        \choice $F(x)=
        \begin{cases}
            ln(x^2+x+1)-1,& x \geq 0,\\
            x^2+x-1,& x < 0.
        \end{cases}$
    \end{choices}
    \question[3]积分$\int_{0}^{1}e^{\frac{-\pi^2}{2}}(1-x^2)dx=$
    \begin{choices}
        \choice $\frac{1}{\sqrt{e}}$;
        \choice $\frac{1}{e}$;
        \choice $e$;
        \choice $\sqrt{e}$.
    \end{choices}
\end{questions}
\section{解答题}
\begin{questions}
    \question[15]设$\lambda$是实数, 使得$f(x)=e^x(x^2-x+\lambda)$的图像的渐近线同时也是曲线$y=f(x)$在某点处的一条切线.
    \begin{parts}
        \part 求$\lambda$的值;
        \part 求$f(x)$的单调区间, 极值和最值;
        \part 求$f(x)$的凹凸性区间, 以及拐点 (写出拐点横坐标即可).
    \end{parts}
    \question[10]已知在平面直角坐标系中, 区域$D$由$x$轴与参数曲线$x(t)=t+arctan t, y(t)=4t(1-t), (0\leq t\leq 1)$共同围成.
    \begin{parts}
        \part 求$D$的面积;
        \part 求图形$D$绕$x$轴旋转一周所得旋转体体积.
    \end{parts}
    \question[10]求二阶线性Euler方程$x^2y^{''}-5xy^{'}+9y=x^3lnx (x>0)$的通解.
    \question[5]设$f(x)$在区间$[0,1]$上连续可微, 满足$0<f^{'}(x)<1, \forall x \in [0,1]$且$f(0)=0$.\\证明: $\int_{0}^{1}(f(x))^3dx\leq (\int_{0}^{1}f(x)dx)^2$.
\end{questions}
\end{document}